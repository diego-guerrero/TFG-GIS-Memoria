\section{Metodología: proceso \textit{software}}
\label{sec:metodologia}

En el ámbito de la ingeniería informática ---y, particularmente, en la ingeniería del \textit{software}---, existe una gran variedad de técnicas, metodologías y \textit{frameworks} que permiten a los equipos de desarrollo organizarse de forma correcta, dividir adecuadamente la carga de trabajo pendiente y planificar temporalmente los hitos a cumplir para alcanzar un método de funcionamiento que optimice al máximo la productividad.

Muestra de ello son los conocidos como ``\textit{frameworks} ágiles'' (del inglés, \textit{agile}), que son un conjunto de metodologías y formas de proceder que destacan por la importancia que otorgan al factor humano, a la capacidad de respuesta frente al cambio o a la colaboración constante con el destinatario o el adquiriente del \textit{software} desarrollado, tal como se detalla en el \textit{Manifiesto Ágil} \cite{Met_AgileManifesto}. Algunos de los \textit{frameworks} ágiles más populares en la actualidad son \textit{Kanban}, \textit{Scrum} o \textit{eXtreme Programming} (XP) \cite{Met_AgileFrameworks}, entre otros.

Sin embargo, estos procesos \textit{software} están concebidos para ser aplicados en grupos de trabajo formados por múltiples integrantes, por lo que no tiene cabida la aplicación directa y fidedigna de sus herramientas en proyectos de un solo autor, como es el caso del presente Trabajo Fin de Grado. Sin embargo, esto no es óbice para poder tomar algunas de las técnicas que introducen y adaptarlas para su utilización y que, de ese modo, poder articular un proceso \textit{software} adecuado para el contexto concreto del presente Trabajo.

El proceso de desarrollo ejecutado toma como pilar esencial uno de los puntos principales de la filosofía ágil: desarrollo iterativo e incremental basado en la consecución de pequeñas iteraciones completas, introduciendo en cada una de estas todos los puntos esenciales que forman parte de los procesos de desarrollo \textit{software}: especificación de  requisitos, diseño y planteamiento de la solución, implementación en código y verificación o \textit{testing}, alcanzando cada uno de ellos la duración temporal requerida para garantizar la completitud de la ejecución del proceso. En el tiempo empleado para completar el Trabajo Fin de Grado se han llevado a cabo un total de 52 iteraciones.

Antes de comenzar el proceso de desarrollo, dada la existencia previa de \textit{software} funcional sobre el que continuar trabajando para la consecución de los objetivos estipulados, fue necesario comenzar con un periodo de aprendizaje y asimilación de la arquitectura de \textit{VSCode4Teaching} y las tecnologías que lo conforman, iniciando la implementación de algunas tareas más sencillas en paralelo y avanzando paulatinamente hacia la ejecución de tareas más complejas a medida que evolucionaba el conocimiento y aprendizaje acerca del proyecto en su completitud.

No obstante, cabe reseñar que no todas las iteraciones realizadas durante el presente proyecto han tenido como finalidad la implementación específica de mejoras o nuevas funcionalidades, sino que algunas de las iteraciones previamente citadas han sido dedicadas a otros tipos de tareas. A este respecto, un ejemplo reseñable son las iteraciones dedicadas a la garantía de seguridad informática, actualizando las dependencias y bibliotecas empleadas en cada uno de los componentes de la aplicación para utilizar versiones revisadas frente a vulnerabilidades de reciente aparición, garantizando así la ausencia de riesgos de seguridad en el \textit{software} desarrollado y ejerciendo un mantenimiento preventivo sobre el proyecto.

Otro caso paradigmático de este hecho son las iteraciones centradas en la corrección de errores o \textit{bugs} detectados y registrados en paralelo al proceso habitual de desarrollo o en pruebas de simulación de entornos reales ---ver \referenciaSeccion{subsec:pruebasManuales}---. En estos casos, se ha seguido un proceso cíclico de tres fases: triaje del error, obteniendo la respuesta a cómo, cuándo, dónde y por qué se produce el error, acotando así su contexto; implementación de la solución, ejecutando las acciones pertinentes sobre el código para solucionar el fallo analizado; y verificación de la solución, comprobando si se sigue produciendo el error y, en caso de ser así, volviendo a ejecutar el proceso íntegro sobre él de nuevo hasta subsanarlo.

La metodología empleada durante este Trabajo Fin de Grado se ha visto soportada en el uso de diversas herramientas de ayuda a la organización de trabajos en materia de programación informática. De este modo, se han empleado recursos como un sistema de control de versiones git o un tablero Kanban en Trello, tal como queda explicado en la \referenciaSeccion{sec:herramientas}.

El tablero de Trello se ha utilizado de la misma forma en que se emplea en la metodología ágil Kanban \cite{Met_TableroKanban}, estructurándolo en las siguientes columnas: \textit{backlog}, que contiene todas las tarjetas pendientes de planificación; \textit{to do}, que incluye todas las tareas que se encuentran inminentes a su comienzo; \textit{doing}, que contiene las tarjetas que se están realizando en el momento de observar el tablero; y \textit{done}, que alberga las tarjetas ya finalizadas en orden de ejecución. De este modo, estos ítems se desplazan entre las columnas anteriormente detalladas consecutivamente y en orden, convirtiendo las tarjetas del \textit{backlog} en tareas atómicas al acceder a \textit{to do}, trasladándolas a \textit{doing} en el momento de comenzar su implementación; y desplazándolas finalmente a \textit{done} una vez finalizada su implementación.
