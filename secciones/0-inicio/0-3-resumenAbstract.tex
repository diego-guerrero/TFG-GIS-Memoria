\chapter*{Resumen}

\textit{VSCode4Teaching} es una extensión para el entorno de desarrollo integrado Visual Studio Code que tiene como objetivo facilitar y potenciar la enseñanza de la disciplina de la programación, contribuyendo así a la mejora de la educación en competencias digitales y en el ámbito de la informática, área que vive un incipiente crecimiento causado por su veloz y revolucionaria implantación y extensión a nivel universal.

En \textit{VSCode4Teaching}, los docentes pueden crear, mantener y supervisar cursos compuestos por ejercicios que los estudiantes matriculados completarán para aprender. Para ello, los profesores proponen una plantilla inicial a cada ejercicio sobre la que los estudiantes construyen la solución que consideren válida, sincronizándola durante su realización para mantener al profesor informado en tiempo real hasta finalizarla.

El presente documento describe de forma pormenorizada todas las cuestiones relativas al tercer trabajo de evolución y adaptación realizado sobre \textit{VSCode4Teaching}, que introduce nuevas características y refina la funcionalidad de esta herramienta para aumentar su alcance y potenciar su usabilidad y mantenibilidad.

Este proyecto tiene una fisonomía de aplicación web compuesta de tres componentes: un servidor, encargado del intercambio, persistencia e interpretación de los datos, una extensión para Visual Studio Code y una aplicación web que permite ampliar la funcionalidad de la aplicación más allá del editor de código.

\textit{VSCode4Teaching} es un proyecto de \textit{software} libre sujeto a la licencia Apache 2.0 a través de un repositorio público de GitHub\footnote{Repositorio: \href{https://github.com/codeurjc-students/2019-VSCode4Teaching}{https://github.com/codeurjc-students/2019-VSCode4Teaching}.} que contiene, además, la documentación necesaria para que otros desarrolladores puedan ejecutar y desplegar la aplicación, pudiendo adaptarla a sus intereses.

\vspace{2pt}

\noindent \textbf{Palabras clave}: educación, informática, programación, desarrollo de aplicaciones web, evolución del \textit{software}, mantenimiento \textit{software}.


\chapter*{Abstract}

\textit{VSCode4Teaching} is an extension for the Visual Studio Code integrated development environment which aims to facilitate and enhance the teaching of programming discipline, thereby contributing to the improvement of education in digital competencies and in the field of computer science, an area experiencing burgeoning growth due to its rapid and revolutionary universal adoption and expansion.

In \textit{VSCode4Teaching}, teachers can create, maintain and supervise courses comprised of exercises that enrolled students will complete for learning purposes. To facilitate this, instructors propose an initial template for each exercise on which students construct their own valid solutions, synchronizing their progress to keep teacher informed in real-time until the exercise is finished.

This document comprehensively describes all matters relating to the third evolution and adaptation work carried out on \textit{VSCode4Teaching}, which introduces new features and refines the functionality of this tool to broaden its scope and enhance its usability and maintainability.

This project has a web application physiognomy built up with three components: a server responsible for data exchange, persistence and interpretation, an extension for Visual Studio Code and a web application that extends the functionality of the application beyond the code editor.

\textit{VSCode4Teaching} is a free \textit{software} project subject to the Apache 2.0 license, hosted on a public GitHub repository\footnote{Repository: \href{https://github.com/codeurjc-students/2019-VSCode4Teaching}{https://github.com/codeurjc-students/2019-VSCode4Teaching}.}, which also contains the necessary documentation for other developers to run and deploy the application, allowing for adaptation to their own interests.

\vspace{2pt}

\noindent \textbf{Keywords}: education, computer science, programming, web applications development, \textit{software} evolution, \textit{software} maintenance.