\chapter{Conclusiones y trabajos futuros}
\label{cap:conclusiones}

Esta sección, última del presente documento, recoge las conclusiones extraídas acerca del trabajo realizado, revisando el cumplimiento de los objetivos estipulados ---\referenciaSeccion{subsec:cumplimientoObjetivos}---, analizando los potenciales trabajos futuros y nuevos objetivos existentes en el proyecto \textit{VSCode4Teaching} ---\referenciaSeccion{subsec:trabajosFuturos}--- e introduciendo una opinión final acerca del transcurso del Trabajo Fin de Grado ---\referenciaSeccion{subsec:aprendizajesPersonales}---.

\section{Cumplimiento de los objetivos estipulados}
\label{subsec:cumplimientoObjetivos}
Respecto a los objetivos del Trabajo Fin de Grado, que quedaron establecidos en el \referenciaCapitulo{cap:objetivos}, y considerando los requisitos especificados en la \referenciaSeccion{sec:requisitos} y desarrollados en las subsiguientes secciones, se constata el cumplimiento de la totalidad de los objetivos prefijados, tal como se desarrolla a continuación:
% . Se revisitan a continuación estos objetivos y se asocia su cumplimiento a los requisitos ejecutados con este fin:
\begin{enumerate}
    \item El objetivo primero establece la necesidad de anonimato de los estudiantes frente al profesorado, requiriendo una modificación del sistema de almacenamiento y la generación de una nueva función que permitiese a los docentes escoger cuándo poder visualizar o no las identidades de sus estudiantes a conveniencia (\referenciaConTT{subsec:rf3}{RF-3}).
    \item El segundo objetivo especifica algunas mejoras acerca de la interacción del profesorado con la aplicación mediante novedades como el sistema de invitación a nuevos docentes (\referenciaConTT{subsec:rf1}{RF-1}) o la posibilidad de añadir varios ejercicios en una única ejecución de un proceso de negocio (\referenciaConTT{subsec:rf2}{RF-2}).
    \item El tercer objetivo estipula la adición de la capacidad para incluir propuestas de soluciones del profesorado en los ejercicios creados. Para ello, se establece que los docentes deben poder añadir soluciones (\referenciaConTT{subsec:rf5}{RF-5}) y controlar, además, cuándo esta solución está disponible a los estudiantes o si pueden continuar editando los ejercicios una vez descargada la solución (\referenciaConTT{subsec:rf6}{RF-6}). Por otro lado, estudiantes y profesores deben poder descargar esta solución, aunque para los estudiantes es preciso controlar que esté disponible (\referenciaConTT{subsec:rf7}{RF-7}). Estos pueden, además, una vez descargada, visualizar fácilmente las diferencias que existan entre ella y su propia propuesta de resolución (\referenciaConTT{subsec:rf8}{RF-8}).
    \item El objetivo cuarto marca la necesidad de potenciar la interacción de los usuarios con la aplicación y se vertebra en dos ejes principales: la necesidad de potenciar la GUI para obtener una apariencia que permita alcanzar una mayor fluidez en el uso de la aplicación y en la introducción de mecanismos de ayuda personalizados para cada usuario con el fin de proporcionar una asistencia eficaz en la utilización de la extensión. En este sentido, se observa el cumplimiento de ambos puntos de la siguiente forma:
    \begin{itemize}
        \item Para mejorar la interfaz de usuario, se introducen modificaciones como: el rediseño integral del \textit{dashboard}, dotándole de una nueva disposición gráfica de elementos y de mayor cantidad de datos actualizados en tiempo real para aportar mejor información al docente (\referenciaConTT{subsec:rf12}{RF-12}); añadiendo, además, la posibilidad de que los docentes previsualicen el \textit{dashboard} de un ejercicio sin necesidad de descargar todos sus ficheros asociados (\referenciaConTT{subsec:rf4}{RF-4}); la incorporación de iconos de colores a la barra lateral para estudiantes y profesores, mostrando más información sobre el estado de cada ejercicio en un solo golpe de vista (\referenciaConTT{subsec:rf11}{RF-11}); y la introducción de un menú contextual ejecutable sobre los cursos para facilitar la ejecución de sus procesos de negocio asociados mediante descripciones textuales aclaratorias, además de mediante los iconos disponibles (\referenciaConTT{subsec:rf10}{RF-10}).
        \item Para proporcionar nuevos mecanismos de ayuda personalizados, se han ejecutado dos actuaciones: la creación de una nueva página de ayuda para estudiantes, facilitando el proceso de inscripción en los cursos mediante explicaciones basadas en el uso de imágenes animadas (\referenciaConTT{subsec:rf9}{RF-9}) y, además, la incorporación de un mecanismo de ayuda rápida en el nuevo \textit{dashboard} para profesores (\referenciaConTT{subsec:rf12}{RF-12}).
    \end{itemize}
    \item El objetivo quinto sitúa la seguridad informática como uno de los pilares fundamentales de esta aplicación. En consonancia con este hecho, se ha ejecutado en varias ocasiones el requisito \referenciaConTT{subsec:rn1}{RN-1} para subsanar las diversas vulnerabilidades aparecidas durante el desarrollo. Además, otros requisitos como el \referenciaConTT{subsec:rf1}{RF-1}, acerca de la reingeniería del proceso para la invitación de nuevos docentes, o algunos de los errores corregidos, analizados en la \referenciaSeccion{subsec:reqsErrores}, eliminan posibles riesgos de seguridad, tales como el \texttt{RE-5}.
    \item El sexto objetivo atribuye una importancia crucial a la detección y corrección de errores durante el proceso de desarrollo o las pruebas en entornos reales. Como consecuencia de la labor de triaje y eliminación de errores surgen todos los requisitos en materia de corrección de errores que se desarrollan en la \referenciaSeccion{subsec:reqsErrores}. Esta labor resulta más sencilla gracias a la ejecución del requisito \referenciaConTT{subsec:rn2}{RN-2}, por el que se han mejorado los sistemas de registro de eventos en todos los componentes de la aplicación, facilitando así el análisis de su comportamiento y las interacciones existentes entre ellos.
    \item El séptimo y último objetivo establece la necesidad de preservar o mejorar la calidad del \textit{software} y atributos relacionados como, por ejemplo, la mantenibilidad. Para ejecutar este objetivo, se han introducido algunas mejoras en el despliegue del servidor (\referenciaConTT{subsec:rn3}{RN-3}) y el requisito \texttt{RN-5}, por el que se ha analizado el código del servidor y la extensión, eliminando duplicidades de código y documentando los procedimientos y algoritmos más complejos.
\end{enumerate}

\section{Trabajos futuros: continuación del proyecto}
\label{subsec:trabajosFuturos}
La realización del trabajo de evolución y mantenimiento analizado en las secciones precedentes ha permitido a las personas involucradas en la concepción y el desarrollo del proyecto \textit{VSCode4Teaching} detectar varias necesidades o potenciales requisitos de interés para su implementación en posteriores etapas del desarrollo.

Entre estas necesidades, la más destacada es la constatación de que muchos estudiantes prefieren utilizar otros entornos de desarrollo distintos de Visual Studio Code, tal como confirma la \textit{Encuesta sobre el Ecosistema de los Desarrolladores} realizada por JetBrains en su edición de 2021 \cite{Conc_EncuestaJetBrains}. Esta encuesta recoge datos sobre los hábitos en torno a la programación de $32\ 570$ desarrolladores, de los que $4567$ afirman ser estudiantes de esta disciplina. Poniendo el foco en este grupo, se obtienen las cifras de uso de IDE que se reflejan en la \referenciaTabla{tab:usoIDEs}. Esta tabla, que refleja cuántos estudiantes utilizan cada uno de los entornos de desarrollo que muestra ---pudiendo cada uno contestar tantos como utilice---, permite observar que Visual Studio Code es el entorno de desarrollo integrado más empleado por los estudiantes de programación, ya que algo más del 53\% de los estudiantes lo utilizan. Sin embargo, son reseñables también las cifras de uso de los entornos de JetBrains, especialmente IntelliJ IDEA y PyCharm, que alcanzan cuotas de uso de alrededor del 39\% y 32\% del estudiantado encuestado, respectivamente.

\begin{table}
    \caption{Adopción de los distintos entornos de desarrollo integrados por los estudiantes encuestados.}
    \label{tab:usoIDEs}
    \centering
    \begin{tabular}{|c|c|c|}
        \textbf{Entorno}   & \textbf{Nº estudiantes} & \textbf{Proporción} \\
        \hline
        Visual Studio Code & 2437                    & 53,36 \%            \\
        \hline
        IntelliJ IDEA      & 1771                    & 38,78 \%            \\
        \hline
        PyCharm            & 1458                    & 31,92 \%            \\
        \hline
        Visual Studio      & 1112                    & 24,35 \%            \\
        \hline
        Android Studio     & 956                     & 20,93 \%            \\
        \hline
        Notepad++          & 685                     & 15,00 \%            \\
        \hline
        Sublime Text       & 637                     & 13,95 \%            \\
        \hline
        CLion              & 565                     & 12,37 \%            \\
        \hline \hline
        Total              & 4567                    &                     \\
        \hline
    \end{tabular}
\end{table}


Como consecuencia de este hecho, se establece como primer y principal trabajo a futuro ejecutar una reorientación del proyecto con el fin de generar una herramienta distinta a la extensión de Visual Studio Code que permita alcanzar la independencia del entorno de desarrollo integrado y que cuente con las mismas o similares funcionalidades que la actual versión, reseñando como piedra angular de la herramienta su capacidad para el seguimiento en tiempo real de la modificación de ficheros locales, de modo que estudiantes y profesores puedan emplear el entorno de desarrollo de su preferencia, emulando las características de las que dispone actualmente con una herramienta independiente del IDE.

Este nuevo trabajo a futuro se suma a los que forman parte de la hoja de ruta del proyecto \textit{VSCode4Teaching}, entre los que cabe destacar los siguientes:
\begin{itemize}
    \item Añadir la capacidad de calificar las propuestas enviadas por los estudiantes para resolver los ejercicios y permitir al alumnado revisar las calificaciones otorgadas, que pueden ser introducidas de forma manual o calculadas automáticamente mediante estrategias para la detección de plagio.
    \item Modificar el sistema de transmisión de las propuestas de resolución de los estudiantes para basarlo en una estrategia de peticiones pequeñas en cada cambio y no en el envío de la propuesta del estudiante completa en cambio para potenciar la eficiencia de la comunicación entre cliente y servidor.
    \item Añadir un rol nuevo para administradores y dotarles de la capacidad para la gestión integral de los usuarios registrados ---estudiantes y profesores--- y los cursos y ejercicios que los componen.
    \item Incorporar nuevas capacidades en torno a los ejercicios, tales como poder fijar fechas de finalización para cada uno o determinar el tiempo máximo disponible para que los estudiantes lo finalicen una vez comenzado.
    \item Integrar la aplicación con entornos Moodle \cite{Moodle} y, específicamente, con el Aula Virtual de la Universidad Rey Juan Carlos mediante el uso de un interfaz LTI\footnote{LTI. Siglas de ``interoperabilidad entre herramientas de aprendizaje'' (del inglés \textit{Learning Tools Interoperability}).} \cite{LTI_Spec}.
\end{itemize}

\section{Aprendizajes personales}
\label{subsec:aprendizajesPersonales}
Me tomo la licencia de redactar en primera persona la sección que cierra esta memoria de mi primer Trabajo Fin de Grado, en la que recopilo las conclusiones personales que aprendo tras esta experiencia.

Mi doble grado ha sido, en mi opinión, un conjunto de más de treinta asignaturas con un marcado carácter teórico y, al menos hasta el tercer curso, han tenido la intención principal de inculcar las bases y principios sobre los que se asienta la informática tal como es entendida en la actualidad. Si bien a partir del tercer curso las asignaturas empiezan a tomar un carácter práctico algo más cercano a la futura realidad laboral que la Universidad busca preparar, este Trabajo Fin de Grado ha sido mi primera oportunidad para ``aterrizar'' todos los conocimientos aprendidos durante el transcurso de la carrera en el contexto de un proyecto real.

Una de las mayores lecciones que aprendo es que la evolución y adaptación del \textit{software} es una de las tareas más complejas que hay en la ingeniería del \textit{software}. Son pocos los proyectos que se arrancan desde cero en un contexto en el que la informática lleva ya al menos veinte años copando todas las áreas y formando parte de la inmensa mayoría de los procesos de negocio existentes, lo que hace que, tal como he tenido oportunidad de ratificar en mi experiencia laboral, muchos de los requisitos necesarios se implementen en proyectos ya existentes previamente. Indudablemente, la tarea más compleja que he ejecutado en este Trabajo Fin de Grado ha sido entender qué se había implementado cuando yo llegué, cómo se había diseñado la herramienta, cómo funcionaban las tecnologías sobre las que se asentaba y de qué manera podría ``abrir'' el código sobre mi ``mesa de operaciones'' para comenzar a implementar los nuevos requisitos que aspirábamos a introducir en el proyecto.

Una vez vencida esta dificultad, este TFG me ha permitido asentar los conocimientos de muchas asignaturas, especialmente de aquellas que versan sobre la ingeniería del \textit{software} en sí misma ---con una especial mención a Evolución y Adaptación del \textit{Software}---, las que enseñan acerca del desarrollo web ---con otra especial mención a Desarrollo de Aplicaciones Web--- y, además, todas las asignaturas ``básicas'' sobre programación, orientación a objetos o redes de computadores, que subyacen debajo de las primeras. Además, me ha permitido acercarme a áreas que había explorado poco, que he utilizado más en el ámbito laboral desde entonces y de las que quiero seguir aprendiendo, como son la computación en la nube, con el enorme potencial que hay detrás de Docker, o el conjunto de prácticas que integran \textit{DevOps}, con especial interés por las buenas prácticas en integración, despliegue y entrega continuos.

Con todos estos aprendizajes, \textit{VSCode4Teaching} continuará creciendo en mi segundo Trabajo Fin de Grado.