\chapter{Objetivos}
\label{cap:objetivos}

\textit{VSCode4Teaching} es, como se ha desarrollado en la \referenciaSeccion{sec:historiaProyecto}, un proyecto ya existente con una amplia funcionalidad disponible para sus usuarios. En ella, los docentes pueden crear cursos con ejercicios de programación basados en plantillas que los alumnos descargarán y sobre las que realizarán sus propuestas de resolución de la problemática planteada, sincronizando su progreso durante su realización, pudiendo así mantener informados en tiempo real a los docentes acerca del progreso de los estudiantes. Durante su construcción y primeros usos, han ido surgiendo de forma progresiva nuevas ideas y requisitos para implementar con el fin de aumentar y mejorar la funcionalidad de la aplicación.

Dada la naturaleza evolutiva y adaptativa de este Trabajo Fin de Grado, el \textbf{principal objetivo} que se marca es alcanzar una evolución del proyecto a través del mantenimiento de su \textit{software}, centrándose en sus vertientes correctiva, implementando soluciones para los errores descubiertos; perfectiva, introduciendo funcionalidad basada en nuevos requisitos y el refinamiento de los actualmente existentes; y preventiva, ejecutando actuaciones para paliar la degradación constante del \textit{software} y conservar su calidad, su eficiencia y su usabilidad.

Tomando como base los puntos de mejora generales existentes en \textit{VSCode4Teaching}, se introduce a continuación la lista de objetivos que se busca ejecutar sobre la herramienta, tanto para crear o refinar funcionalidades y procesos de negocio como para, además, mejorar el propio \textit{software} del proyecto. Estos objetivos, que son el eje vertebrador del Trabajo Fin de Grado, son:

\begin{enumerate}
    \item \underline{Anonimato de los estudiantes}: modificar el sistema de almacenamiento de propuestas de resolución de ejercicios del estudiantado para designarlas de forma anónima, permitiendo al profesorado distinguirlas mediante una relación entre el nombre de cada estudiante y el del directorio anónimo correspondientemente asignado.
    \item \underline{Mejor funcionalidad para docentes}: introducir nuevas funcionalidades para facilitar la interacción de los profesores con la aplicación, como la creación de un nuevo sistema de registro mediante invitación de otros docentes o la incorporación de una característica para que los profesores puedan añadir ejercicios de forma múltiple a sus cursos, entre otros puntos.
    \item \underline{Publicación de propuestas de solución de ejercicios}: aumentar la funcionalidad asociada a los ejercicios, haciendo que puedan albergar una propuesta de solución elaborada por el profesor, permitiéndole añadirla opcionalmente e incorporando los controles pertinentes para poder ajustar su disponibilidad hacia los estudiantes.
    \item \underline{Mejor interfaz de usuario}: potenciar la experiencia de usuario de todos los usuarios poniendo el foco en la obtención de una mejor GUI\footnote{GUI. Siglas de ``interfaz guiada de usuario'' (del inglés \textit{Guided User Interface}). Es el conjunto de elementos gráficos que integran la interfaz empleada por las herramientas informáticas para interactuar bidireccionalmente con los usuarios.} que, mediante nuevos elementos gráficos, facilite el uso de la aplicación, haciéndola más intuitiva y sencilla. Además, con el mismo propósito, incorporar mejoras como la generación de nuevos mecanismos de ayuda personalizados y adecuados a las necesidades específicas de cada alumno y profesor.
    \item \underline{Ciberseguridad}: garantizar el máximo grado de seguridad informática posible para todos los usuarios de \textit{VSCode4Teaching}, introduciendo nuevas políticas de seguridad y manteniendo actualizadas las contramedidas implementadas frente a la aparición de nuevas vulnerabilidades.
    \item \underline{Verificación del \textit{software}}: introducir más pruebas en la aplicación para lograr detectar la mayor cantidad posible de \textit{bugs}\footnote{\textit{Bug}. Es un error producido en el proceso de codificación que provoca incoherencias o fallos en el uso de la aplicación.} o errores presentes en el \textit{software} para, de este modo, proceder a su diagnóstico y corrección.
    \item \underline{Calidad del proyecto y de su \textit{software}}: incorporar mejoras específicamente orientadas a la calidad del \textit{software} y generar una mejor documentación del programa y sus distintos componentes, facilitando así el posterior desarrollo, evolución, mantenimiento y despliegue de la aplicación a otros usuarios y desarrolladores.
\end{enumerate}
