\subsubsection{\texttt{RN-2}: sistemas de registro de eventos}
\label{subsec:rn2}

La mantenibilidad es uno de los atributos de calidad esenciales en todos los proyectos \textit{software}, especialmente en aquellos con naturaleza de código libre como \textit{VSCode4Teaching}. Es por ello que resulta crucial introducir requisitos que permitan dar una evolución y mantenimiento adecuados a las aplicaciones para poder ejecutar desarrollos posteriores de forma más sencilla y rápida. Una de estas herramientas son los sistemas de registro de eventos o \textit{logs} y que permiten a los desarrolladores depurar y comprender el funcionamiento de las aplicaciones de forma más adecuada en tiempo real mientras funcionan, dejando constancia escrita de aquellas acciones que se deseen registrar a medida que suceden durante el uso de la aplicación.

\textit{VSCode4Teaching} introduce sistemas de registro de eventos en sus tres componentes, potenciando el uso de los anteriormente existentes e incorporando nuevas librerías que permiten organizarlos de forma más adecuada. El servidor introduce la biblioteca \textit{Log4J} \cite{rn2_tecLogServidor}, que está integrada dentro del \textit{framework} Spring en el que se basa ---tal como queda reflejado en la \referenciaSeccion{subsec:tecServidor}---, potenciando su utilización mediante la revisión de los eventos inventariados anteriormente existentes y la introducción de registros de actividades como la recepción y gestión de peticiones y el envío de sus respectivas respuestas.

Por otro lado, la extensión introduce la librería \textit{Winston} \cite{rn2_tecLogExtension} con el mismo propósito, permitiendo generar una traza de las peticiones que, cotejándolo con el registro anteriormente comentado, permite analizar su interacción con el servidor, recogiendo las peticiones enviadas y recibidas, las demoras temporales que se producen o el estado del cuerpo de las peticiones en su envío y en su recepción.

Asimismo, aunque su utilización es menor, también se introduce en la aplicación web SPA la librería \textit{ngx-logger} \cite{rn2_tecLogAppWeb} y se reflejan registros de las peticiones que envía al servidor durante los procesos de usuario ejecutados en este componente.
