\subsubsection{\texttt{RN-1}: ciberseguridad y mantenimiento preventivo}
\label{subsec:rn1}

La ciberseguridad es la ``práctica de proteger sistemas, redes y programas de ataques digitales [...] que apuntan a acceder, modificar o destruir la información confidencial, extorsionar a los usuarios o interrumpir la continuidad del negocio'' \cite{rn1_queEsCiber}.

De esta definición se extrapola el papel fundamental que la seguridad informática ocupa en el desarrollo y mantenimiento de todos los programas, generando un trabajo constante y necesario de detección y solución de problemas en materia de confidencialidad, integridad y disponibilidad ---los pilares básicos de la seguridad informática--- de las herramientas informáticas y los datos que involucran en su uso.

Durante el desarrollo de este Trabajo Fin de Grado, han sido detectadas numerosas vulnerabilidades en las dependencias y \textit{frameworks} empleados para la implementación de \textit{VSCode4Teaching}, condición suficiente para la introducción de cuatro \textit{sprints} o pequeños periodos de tiempo dedicados específicamente a la actualización de estas librerías a sus últimas versiones disponibles.

El servidor ha sido el más afectado por las vulnerabilidades surgidas en los últimos meses. Alrededor de este componente, que se basa en el uso del \textit{framework} Spring ---véase \referenciaSeccion{subsec:tecServidor} para más información---, han aparecido importantes vulnerabilidades. Algunas de las más destacadas han sido catalogadas en la lista \textit{Common Vulnerabilities and Exposures} (CVE\footnote{CVE. Siglas de ``vulnerabilidades y revelaciones comunes'' (del inglés \textit{Common Vulnerabilities and Exposures}). Es una de las listas de recopilación de vulnerabilidades y riesgos de seguridad más extendidas actualmente.}) \cite{rn1_cve} como: CVE-2021-44228, CVE-2021-45046, CVE-2021-45105 (noviembre-diciembre 2021, conocidas como ``Log4j'') y CVE-2022-22965 (enero 2022, conocida como ``Spring4Shell''). Como consecuencia, se ha introducido una actualización de la versión de Spring Boot a la más reciente disponible en la fecha de cierre del trabajo (2.7.5), así como de las dependencias empleadas para el funcionamiento del servidor.

Asimismo, sobre la extensión para Visual Studio Code se han ejecutado actualizaciones de dependencias y \textit{frameworks} para introducir correcciones a vulnerabilidades como las identificadas como CVE-2021-3918 (noviembre 2021), CVE-2021-44906 (diciembre 2021) o CVE-2022-0155 (enero 2022), así como actualizaciones de las diversas librerías empleadas a las últimas versiones divulgadas.

La aplicación web SPA también ha recibido diversas actualizaciones, aplicadas tanto a su \textit{framework} de base, Angular (actualizado a versión 14.2.11) como a alguna de sus dependencias por motivo de mejora en materia de seguridad y de rendimiento.
