Una vez inician su explotación ---es decir, su utilización habitual por parte de los usuarios---, las aplicaciones informáticas comienzan a registrar errores existentes en el flujo de uso, procedentes tanto de errores en la implementación de alguno de los componentes de la aplicación como de fallos en la comunicación entre ellos.

Tal como se desarrolla en la \referenciaSeccion{sec:metodologia}, la detección y corrección de errores se ha ejecutado paralelamente al desarrollo e incorporación de los demás requisitos descritos con anterioridad, ejecutando iteraciones de diagnóstico y solución de errores en tres fases: triaje del error, realizando una ``anamnesis'' para determinar las condiciones en las que se producía el error ---cuándo se producía, si se derivaba de alguna acción específica de los usuarios de un cierto rol, si sucedía periódicamente o cuáles eran las consecuencias derivadas de su ocurrencia, entre otras, con el fin de recolectar toda la información posible acerca del error---; primera implementación de una corrección, haciendo las modificaciones oportunas en el código fuente de los componentes involucrados para tratar de corregir el error; y verificación de la corrección, reproduciendo las condiciones en las que se produjo el error originalmente y evaluando si se vuelve a producir, si se produce algún otro error derivado de la corrección o si se ha subsanado en su totalidad.

El descubrimiento de nuevos errores para su posterior corrección se ha ejecutado a través de las pruebas realizadas para la verificación del \textit{software}, que se ven descritas en la \referenciaSeccion{sec:verificacion}. Las pruebas implementadas sobre el código ---esto es, las pruebas unitarias y de integración--- han permitido detectar errores lógicos mediante la comparación de resultados esperados frente a resultados obtenidos, dando lugar a algunos requisitos como el \texttt{RE-1}.
Por otro lado, las pruebas de uso en simulaciones de entornos reales han permitido encontrar fallos en la aplicación a través de la imitación de potenciales flujos de uso habituales en entornos en explotación; esto es, recreando acciones y comportamientos de usuarios, encontrando fallos que las pruebas de código implementadas para este proyecto no permiten localizar, como pueden ser los requisitos \texttt{RE-4} o \texttt{RE-6}.

Tal como se introducía en la \referenciaSeccion{subsec:listaReqsErrores}, cabe reseñar seis requisitos de entre todas las correcciones de errores ejecutados en el proyecto \textit{VSCode4Teaching}:
\begin{itemize}
    \item \texttt{\textbf{RE-1}}. Los profesores disponen de una tabla en el \textit{dashboard} para visualizar el progreso individualizado de los estudiantes en la realización de los ejercicios propuestos. Esta tabla tiene controles que permiten la ordenación de sus filas alfabéticamente según los valores de cada una de las columnas. Cuando se pretendía ordenar la tabla del \textit{dashboard} según alguna columna, el resultado de la ordenación mostrado era incorrecto, pudiendo ser ordenados por los valores de columnas diferentes o de forma aparentemente arbitraria.
    \item \texttt{\textbf{RE-2}}. Los profesores tienen a su disposición una funcionalidad para añadir varios ejercicios simultáneamente en un determinado curso. Cuando la empleaban, aparecía un directorio intermedio no existente en origen con el mismo nombre en el que se introducían todos los contenidos del ejercicio, que se mostraban en un nivel superior cuando se agregaban de uno en uno, de modo que el comportamiento de la funcionalidad daba resultados diferentes según si se ejecutaba para ejercicios individuales o para grupos de ejercicios.
    \item \texttt{\textbf{RE-3}}. Los estudiantes pueden tener los ejercicios de los cursos en los que participan en distintos estados: ``no comenzado'', cuando aún no han abierto los ejercicios en su editor de código; ``en progreso'', cuando han accedido al ejercicio en su editor; y ``finalizado'', cuando, una vez han realizado modificaciones, deciden marcarlo como finalizado, bloqueando sucesivas ediciones de su propuesta de resolución del ejercicio. Las pruebas de simulación de entornos reales permitieron detectar algunos errores que impedían que la extensión siguiese adecuadamente el progreso de los estudiantes al no modificar el estado a ``en progreso'' al comenzar el ejercicio, de modo que las modificaciones realizadas no quedaban almacenadas en el servidor. Además, una vez que los estudiantes marcaban como finalizados los ejercicios, en caso de que los descargasen de nuevo, veían revertido su estado a ``en progreso'', permitiéndoles volver a editar sus propuestas de resolución una vez marcadas como finalizadas.
    \item \texttt{\textbf{RE-4}}. En la tabla que se muestra a los profesores en el \textit{dashboard}, uno de los datos que quedan reflejados es la fecha de última modificación realizada por cada uno de los estudiantes. Este dato refleja cuánto tiempo ha transcurrido desde la última modificación registrada en base de datos, actualizándose periódicamente para reflejar valores correctos mientras se mantenga abierta esta visualización. En una prueba de simulación de entorno real se descubrió que una modificación en la interfaz de usuario de esta visualización provocó que la rutina periódica tuviese una configuración inadecuada, readaptándose para su correcta visualización en posteriores ediciones.
    \item \texttt{\textbf{RE-5}}. Cuando se producen cierres de sesión en la extensión de \textit{VSCode4Teaching}, es preciso ocultar y deshabilitar todas las funciones propias de los usuarios autenticados, ya que su utilización posterior al fin de la autenticación puede entrañar un riesgo de seguridad, pues los elementos de la interfaz de usuario que permiten la ejecución de funcionalidades como, por ejemplo, la visualización del \textit{dashboard}, de acceso restringido a profesores, quedaban visibles y funcionales en la GUI. Para eliminar este error, se introducen nuevos controles que dotan a la extensión de capacidades para eliminar todos los elementos de la interfaz que solo pueden ser ejecutados por usuarios identificados en la extensión.
    \item \texttt{\textbf{RE-6}}. Las pruebas realizadas en simulación de entornos reales permiten también detectar problemas de rendimiento y carga de la aplicación, tal como se desarrolla en la \referenciaSeccion{subsec:pruebasManuales}. Uno de los errores producidos como consecuencia de un problema de estas características se manifiesta ante la ejecución de la funcionalidad de la que los profesores disponen para cargar ejercicios de forma masiva en sus cursos. Anteriormente, se enviaban todas las peticiones para la subida de plantillas y propuestas de solución (en caso de haberlas) de forma simultánea desde la petición, ocasionando que el servidor recibiese una alta cantidad de peticiones a la vez, hecho que provocaba el incorrecto procesamiento de los ejercicios. Se ha introducido una modificación para hacer que la extensión envíe las peticiones de forma secuencial, lanzando una nueva petición cada vez que se recibe confirmación del servidor a la inmediatamente anterior, mitigando el problema de carga del servidor.
\end{itemize}
